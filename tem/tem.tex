\documentclass[a4paper,12pt]{article}
\usepackage[utf8]{inputenc}
\usepackage[T1]{fontenc}
\usepackage[hungarian]{babel}
\usepackage{graphicx}
\usepackage{geometry}
\geometry{a4paper,
		     tmargin = 35mm, 
		     lmargin = 25mm,
		     rmargin = 30mm,
		     bmargin = 30mm}
\usepackage{mathtools}
\usepackage{amsmath}
\usepackage{color}
\usepackage{setspace}
\usepackage{amsmath,amssymb}
\usepackage{float}
\usepackage{hyperref}

\usepackage{indentfirst}
\usepackage{subfig}

\usepackage{siunitx}

\renewcommand\thesection{\Roman{section}}

\begin{document}

\linespread{1.25}

\begin{titlepage}

	\centering
	\includegraphics[width=0.66\textwidth]{elte.jpg}\par\vspace{1cm}
	{\scshape\LARGE ELTE TTK \par}
	\vspace{3cm}
	{\scshape\Large Elektronmikroszkópia \par}
	\vspace{1cm}
	{\large\itshape Olar Alex\par}
	\vspace{3cm}
	{\large 2018 \par}
	
\end{titlepage}

\tableofcontents

\newpage

\section{Elméleti összefoglaló, mérési eszközök}

\vspace{5mm}

\par A mérés során egy transzmissziós elektronmikroszkópot használtunk, mellyel különböző mintákat vizsgáltunk meg. A feladatunk a mikroszkóp kameraállandójának meghatározása volt, majd ezután egy $Si$ mintán végeztük diffrakciós mérést.

\vspace{5mm}

\par A mérés során a képeket 'image plate'-re rögzítettük, amiket előhívás után, elektronikus formában megkaptunk.

\section{Kalibráció}

\vspace{5mm}

\par Polikristályos nikkelt használva a klaibráláshoz igen egyszerű összefüggést kapunk köbös rácsra

\vspace{5mm}

\begin{equation}
R_{hkl} = \frac{L\lambda}{a}\sqrt{h^{2} + k^{2} + l^{2}}
\end{equation}

\vspace{5mm}

\par Ahol $a$ a rácsállandó, $\lambda$ az elektron hullámhossz. Természetesen ez még ennél is egyszerűbb hiszen $d = \frac{a}{\sqrt{h^{2} + k^{2} + l^{2}}}$, ami meg van adva a \url{http://www.energia.mta.hu/~labar/Ni_cF4_04-010-6148.pdf} alatt.

\vspace{5mm}

\begin{figure}[!htb]
\centering
\includegraphics[width=0.55\linewidth]{./Ni-Calibration.jpg}
\caption{ A nikkel gyűrűs diffrakciós képe, ami a polikristályos elrendeződés miatt alakul ki. A kalibrációhoz függőlegesen a 754. pixelnél vettem ki egy oszlopot}
\end{figure}

\vspace{5mm}

\par A kalibrációhoz használt intenzitás csúcsokat ábrázolva

\vspace{5mm}

\begin{figure}[!htb]
\centering
\includegraphics[width=0.65\linewidth]{./kalib.png}
\caption{ Az intenzitás (255 - intenzitásként) van ábrázolva, hogy a fekete gyűrűk legyenek a csúcsok}
\end{figure}

\vspace{5mm}

\par A csúcsokra Gauss-függvényeket illesztettem konstans háttérrel. A középső foltra nem illesztettem, csak az azt határoló 8 csúcsra. Ebből kaptam 4 csúcsot, melyek rendre

\vspace{5mm}

\begin{center}
\begin{tabular}{|c|c|}
\hline
Csúcs helye [$pixel$] & $\Delta$ csúcs helye [$pixel$] \\
\hline
257.74 & 0.11 \\
\hline
308.49 & 0.01 \\
\hline
395.56 & 0.05 \\
\hline
422.93 & 0.11 \\
\hline
784.17 & 0.07 \\
\hline
812.01 & 0.09 \\
\hline
899.2 & 0.07 \\
\hline
950.06 & 0.05 \\
\hline
\end{tabular}
\end{center}

\vspace{5mm}

\par Ebből már könnyen számolhatóak a sugarak, mert csak páronként ki kell vonni egymásból a mért értékeket (első négyből a második négyet). A hibát négyzetes hibaterjedéssel számoltam.

\vspace{5mm}

\par Sorrendben ezek a gyűrűk sugarai, a legintenzívebb pontot kihagyva. Az ezekhez tartozó $d$ távolságokat a korábbi linkről véve:

\vspace{5mm}

\begin{center}
\begin{tabular}{|c|c|c|}
\hline
R [$pixel$] & $\Delta$R [$pixel$] & d [\si{\angstrom}] \\
\hline
180.62 & 0.13 & 2.037180\\
\hline
208.23 & 0.10 & 1.764250 \\
\hline
295.36 & 0.07 & 1.247510\\
\hline
346.16 & 0.12 & 1.063880\\
\hline
\end{tabular}
\end{center}

\vspace{5mm}

\par Egyenest illesztve tehát kapjuk a meredekségből a kameraállandót

\vspace{5mm}

\begin{figure}[!htb]
\centering
\includegraphics[width=0.45\linewidth]{./kalib_egyenes.png}
\caption{ Az intenzitás (255 - intenzitásként) van ábrázolva, hogy a fekete gyűrűk legyenek a csúcsok}
\end{figure}

\vspace{5mm}

\begin{equation}
L\lambda = (368.16 \pm 0.21) ~pixel\cdot \si{\angstrom}
\end{equation}

\vspace{5mm}

\section{Egykristrály diffrakció}

\vspace{5mm}

\par A következőkben mindannyian egy $Si$ kristályról készült, különböző állású diffrakciós képet vizsgáltunk. Az én képem a következő volt

\vspace{5mm}

\begin{figure}[!htb]
\centering
\includegraphics[width=0.45\linewidth]{./Si_Olar.jpg}
\caption{ Si egykristály diffrakciós képe }
\end{figure}

\vspace{5mm}

\par Jelölve az ábrán a kiértékelt pontokat, majd a legintezívebb ponttól számítva $K(738, 630) ~[px, px]$ kiértékelve azok távolságát, az indexelést elvégeztem.

\vspace{5mm}

\begin{figure}[!htb]
\centering
\includegraphics[width=0.55\linewidth]{./Si_Olar_betuzott.jpg}
\caption{ Si egykristály diffrakciós képe, a pontok az azonosításhoz betüzve vannak }
\end{figure}

\vspace{5mm}

\begin{center}
\begin{tabular}{|c|c|c|c|c|c|c|c|c|}
\hline
Pont & x [pixel] & y [pixel] & |$\Delta$x| [pixel] & |$\Delta$y| [pixel] & $\Delta$ [pixel] & d [$\si{\angstrom}$] & $\Delta$d [$\si{\angstrom}$] & index \\
\hline
A & 640 & 790 & 98 & 160 & 187.63 & 1.962 & 0.001 & \{220\} \\
\hline
B & 858 & 814 & 120 & 184 & 219.67 & 1.676 & 0.001 & \{311\} \\
\hline
C & 960 & 652 & 222 & 22 & 223.09 & 1.650 & 0.001 & \{311\} \\
\hline
D & 1069 & 490 & 322 & 140 & 351.12 & 1.049 & 0.001 & \{511\} \\
\hline
E & 834 & 466 & 96 & 164 & 190.03 & 1.937 & 0.001 & \{220\} \\
\hline
F & 618 & 444 & 120 & 186 & 221.35 & 1.663 & 0.001 & \{311\} \\
\hline
G & 514 & 608 & 224 & 22 & 225.08 & 1.636 & 0.001 & \{311\} \\
\hline
\end{tabular}
\end{center}

\vspace{5mm}

\par Ez természtesen még nem jó, hiszen így nem adják ki a lineárkombinációk egymást, permutálást és előjelváltást kell végezni a tükrözési és forgatási szimmetriák miátt. Így

\vspace{5mm}

\begin{center}
\begin{tabular}{|c|c|c|c|c|c|c|c|c|}
\hline
Pont & x [pixel] & y [pixel] & d [$\si{\angstrom}$] & index & \text{egy vektoriálisan helyes indexelés} \\
\hline
A & 640 & 790 &  1.962 &  \{220\} & \{2,2,0\} \\
\hline
B & 858 & 814  & 1.676 & \{311\} & \{-1,3,-1\} \\
\hline
C & 960 & 652 & 1.65 & \{311\} & \{-3,1,-1\}\\
\hline
D & 1069 & 490 & 1.0485 & \{511\} & \{-5,-1,-1\}\\
\hline
E & 834 & 466 & 1.937 & \{220\} & \{-2,-2,0\} \\
\hline
F & 618 & 444 & 1.663 & \{131\} & \{1,-3,1\} \\
\hline
G & 514 & 608 & 1.636 & \{311\} & \{3,-3,1\} \\
\hline
\end{tabular}
\end{center}

\vspace{5mm}

\par Az indexelés során $A$-t tetszőlegesen választhattam így azt meghagytam a kezdeti indexelés szerint. A $B$-t úgy válaszottam meg, hogy a $(-1)$-szeresével kiadja a $C$-t, ami olyan típusú, mint az $B$ csak esetleg permutálva, előjelcserével. Ha helyesen jártam el, akkor meg kellett kapjam a $C$-t és az $E$-t amelyek 'vektoriális' összegének ki kellett adnia a $D$-t. Ez ellenőrizhető.

\vspace{5mm}

\par A labor alatt megbeszéltek alapján a zónatengely a két bázisvektor, jelen esetben $A$-hoz és $B$-hez tartozó indexek, de ugyan így lehetne akár $C$ és $E$ is, vektoriális szorzata.

\begin{equation*}
\vec{A} \times \vec{B} = (2,2,0) \times (-1,3,-1) = (-2, 2, 8) \quad \tilde \quad \quad (-1, 4, 4)
\end{equation*}

\section{Összegzés}

\par A harmadik fejezetben a leolvasás során vétettem hibát termszetesen, ezért nem túl pontosan illeszkednek a $d$ távolságok a \url{http://www.energia.mta.hu/~labar/Si_cF8_04-002-0118.pdf} alatt találhatóakhoz, de nagyjából be lehetett lőni azokat. A legnagyobb bizonytalanságot a $D$ pont jelentette, mert az közelebb volt a (4,2,2)-es index hármashoz, de a többi pont alapján egyértelműen nem annak kellett lennie. Összességénen sikeresnek tekinthető a mérésem.

\end{document}