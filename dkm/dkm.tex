\documentclass[a4paper,12pt]{article}
\usepackage[utf8]{inputenc}
\usepackage[T1]{fontenc}
\usepackage[hungarian]{babel}
\usepackage{graphicx}
\usepackage{geometry}
\geometry{a4paper,
		     tmargin = 35mm, 
		     lmargin = 25mm,
		     rmargin = 30mm,
		     bmargin = 30mm}
\usepackage{mathtools}
\usepackage{amsmath}
\usepackage{color}
\usepackage{setspace}
\usepackage{amsmath,amssymb}
\usepackage{float}
\usepackage{hyperref}

\usepackage{indentfirst}
\usepackage{subfig}

\usepackage{siunitx}

\renewcommand\thesection{\Roman{section}}

\begin{document}

\linespread{1.25}

\begin{titlepage}

	\centering
	\includegraphics[width=0.66\textwidth]{elte.jpg}\par\vspace{1cm}
	{\scshape\LARGE ELTE TTK \par}
	\vspace{3cm}
	{\scshape\Large Dinamikus nano- és mikrokeménység mérése \par}
	\vspace{1cm}
	{\large\itshape Olar Alex\par}
	\vspace{3cm}
	{\large 2018 \par}
	
\end{titlepage}

\tableofcontents

\newpage

\section{Elméleti összefoglaló, mérési eszközök}

\par A mérés során egy keménységmérő eszközt ismertünk meg, amely Vickers-fejjel végezte el a méréseket. Ezek során tiszta anyagok ($Ni, Cu, Al, Ag$) keménységét mértük meg, valamint $Al, Mg$ különböző ötvözeteit vizsgáltuk. Feladatunk volt még a plasztikus instabilitás kvalitatív vizsgálata is a kiértékelés utolsó részében.

\section{Kiértékelés}

\par A Vickers-fej mellett a minta keménysége az alkalmazott erő és a hatékony felület hányadosa, $HV = \frac{F}{A}$. A mérés során dinamikus mérést végzünk, hiszen egy $F-h$, azaz erő-benyomódás görbét vizsgálunk. A görbe alatti terület lehetőséget ad a disszipált energia kiszámítására, valamint a görbéből folyáshatárra, Young-modulusra és a mért anyagok egyéb rugalmas tulajdonságaira következtethetünk. 

\vspace{5mm}

\par A maximális erőhöz tartozó benyomódás $h_{max}$ szükséges a további számolásokhoz. A szükséges korrigált mélységet az alábbi egyenlet adja:

\begin{equation*}
h_{c} = h_{m} - 0.75\frac{F_{m}}{\frac{dF}{dh}|_{h_{m}}}
\end{equation*}

\par , amit azért kell alkalmazni, mert a statikus és dinamikus esetben a benyomódás eltér és a következő korrekció szükséges annak visszanyeréséhez. 

\par A kiértékelés során a terheletlen szakaszra egyenest illesztettünk és annak meredekségét használtuk $\frac{dF_{m}}{dh}|_{h_{m}}$ kiszámításához. Míg $F_{max}$-ot az adatsorból meghatározva a hozzá tartozó $h_{max}$-al automatikusan adott volt. Így:

\begin{center}
\begin{tabular}{|c|c|c|}
\hline
Anyag & $h_{max} [\mu]$ & $F_{max} [mN]$ \\
\hline
Al &5.186 &0.191 \\
\hline
Cu &2.233 &0.192 \\
\hline
Ni &1.604 &0.192 \\
\hline
Ag& 3.298& 0.191 \\
\hline
acél &2.364& 0.191 \\
\hline
Al - 0.47$\%$ Mg &  4.156& 0.192 \\
\hline
Al - 0.93$\%$ Mg& 4.056& 0.191 \\
\hline
Al - 1.25$\%$ Mg& 3.743& 0.192 \\
\hline
Al - 1.45$\%$ Mg &3.481& 0.191 \\
\hline
Al - 2.7$\%$ Mg &3.416& 0.191 \\
\hline
Al - 4.5$\%$ Mg &2.878& 0.191 \\
\hline
Al - 7.3$\%$ Mg &2.484& 0.192 \\
\hline
\end{tabular}
\end{center}

\par $h_{max}$ hibáját $0.05 ~\mu m$-re becsültem, mivel ez sokkal nagyobb volt, mint $h_{c}$-nek az illesztésből származó hibája, így $\delta h_{c} = \delta h_{max}$.

\begin{center}
\begin{tabular}{|c|c|c|}
\hline
Anyag & $h_{c} [\mu m]$ & $\Delta h_{C} [\mu m]$ \\
\hline
Al &5.19 &0.05 \\
\hline
Cu &2.24 &0.05 \\
\hline
Ni &1.61 &0.05 \\
\hline
Ag& 3.30& 0.05\\
\hline
acél &2.36& 0.05 \\
\hline
Al - 0.47$\%$ Mg &  4.16& 0.05 \\
\hline
Al - 0.93$\%$ Mg& 4.06& 0.05 \\
\hline
Al - 1.25$\%$ Mg& 3.75& 0.05 \\
\hline
Al - 1.45$\%$ Mg &3.48& 0.05 \\
\hline
Al - 2.7$\%$ Mg &3.42& 0.05 \\
\hline
Al - 4.5$\%$ Mg &2.88& 0.05 \\
\hline
Al - 7.3$\%$ Mg &2.48& 0.05 \\
\hline
\end{tabular}
\end{center}

\par Jól látható, hogy a korrekció olyan kicsi, hogy a benyomódás csak nagyon kis mértéken belül változott.

\par A Vickers-fej tulajdonsága, hogy az érintkező felület pedig:

\begin{equation*}
A = 24.5h_{c}^{2}
\end{equation*}

\par A felületek kiszámolva $h_{c}$-ből:

\begin{center}
\begin{tabular}{|c|c|c|}
\hline
Anyag & $A [\mu m^{2}]$ & $\Delta A [\mu m^{2}]$ \\
\hline
Al &659.159 &0.0001\\
\hline
Cu &122.16 &0.0001\\
\hline
Ni& 63.024 &0.0001\\
\hline
Ag &266.478 &0.0001\\
\hline
acél& 137.393 &0.0002\\
\hline
Al - 0.47$\%$ Mg &423.219 &0.0001\\
\hline
Al - 0.93$\%$ Mg &403.038 &0.0001\\
\hline
Al - 1.25$\%$ Mg &343.334 &0.0001\\
\hline
Al - 1.45$\%$ Mg &297.045 &0.0001\\
\hline
Al - 2.7$\%$ Mg &285.921 &0.0001\\
\hline
Al - 4.5$\%$ Mg &203.027 &0.0001\\
\hline
Al - 7.3$\%$ Mg &151.155 &0.0001\\
\hline
\end{tabular}
\end{center}

\par A redukált modulus $E_{r}$ egyből számolható a korábbiak ismeretében, ugyanis:

\begin{equation*}
E_{r} = \frac{\sqrt{pi}\frac{dF}{dh}|_{h_{m}}}{2\beta\sqrt{A}}
\end{equation*}

\par Ahol $\beta = 1.012$, és a fentebbire azért van szükség egyáltalán, mivel maga a mérőfej is rugalmas anyag, így deformálódik. Azonban innen, a mért anyag Poisson-számának ismeretében már származtatható annak Young-modulusa.

\begin{equation*}
\frac{1}{E_{r}} = \frac{1 - \nu^{2}}{E} + \frac{1-\nu^{2}_{i}}{E_{i}}
\end{equation*}

\par Ahol $E_{i} = 1070 GPa$ a fej Young-modulusa, $\nu_{i} = 0.17$ szintén a fejre jellemző Poisson-szám.

\par Ebből a megfelelő $\nu$ paramétert helyettesítve már egyből az anyagok Young-modulusát számoltam:

\begin{center}
\begin{tabular}{|c|c|c|}
\hline
Anyag & $E [GPa]$ & $\Delta E [GPa]$ \\
\hline
Al &38.284 &0.247\\
\hline
Cu &108.596 &1.243\\
\hline
Ni &146.456 &1.61\\
\hline
Ag &72.55 &0.997\\
\hline
acél &59.323 &0.674\\
\hline
Al - 0.47$\%$ Mg &54.592& 1.384\\
\hline
Al - 0.93$\%$ Mg &33.18 &0.272\\
\hline
Al - 1.25$\%$ Mg &65.824& 1.149\\
\hline
Al - 1.45$\%$ Mg & 59.421& 0.71\\
\hline
Al - 2.7$\%$ Mg &76.486& 1.275\\
\hline
Al - 4.5$\%$ Mg &81.084& 0.744\\
\hline
Al - 7.3$\%$ Mg &59.394& 0.363\\
\hline
\end{tabular}
\end{center}

\par A továbbiakban az $Al, Mg$ ötvözetek keménységének meghatározása volt a cél. Erre:

\begin{equation*}
HV = HV_{0} + Bc^{m}
\end{equation*}

\par ahol $m$ kitevő modellfüggő. Ezen kívül még vizsgálnunk kellett a plasztikus instabilitást, melyhez alacsony sebességű benyomásnál az $F-h$ görbe 'fogazottságát' kell figyelmesebben megvizsgálnunk. 

\par A keménységet $\frac{F}{A}$-ból származtatva az összes anyagra:

\begin{center}
\begin{tabular}{|c|c|c|}
\hline
Anyag & $HV [MPa]$ & $\Delta HV [MPa]$ \\
\hline
Al & 290.238 & 2.149\\
\hline
Cu & 1568.648 & 11.579\\
\hline
Ni & 3041.951&  22.48\\
\hline
Ag & 717.532 & 5.307\\
\hline
acél & 1412.266 & 12.865\\
\hline
Al - 0.47$\%$ Mg & 452.789&  3.343\\
\hline
Al - 0.93$\%$ Mg & 473.894 & 3.509\\
\hline
Al - 1.25$\%$ Mg & 559.97&  4.129\\
\hline
Al - 1.45$\%$ Mg & 643.526&  4.77\\
\hline
Al - 2.7$\%$ Mg & 670.051 & 4.951\\
\hline
Al - 4.5$\%$ Mg & 944.04 & 6.978\\
\hline
Al - 7.3$\%$ Mg & 1266.943 & 9.357\\
\hline
\end{tabular}
\end{center}

\end{document}\grid
