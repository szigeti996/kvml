\documentclass[a4paper,12pt]{article}
\usepackage[utf8]{inputenc}
\usepackage[T1]{fontenc}
\usepackage[hungarian]{babel}
\usepackage{graphicx}
\usepackage{geometry}
\geometry{a4paper,
		     tmargin = 35mm, 
		     lmargin = 25mm,
		     rmargin = 30mm,
		     bmargin = 30mm}
\usepackage{mathtools}
\usepackage{amsmath}
\usepackage{color}
\usepackage{setspace}
\usepackage{amsmath,amssymb}
\usepackage{float}


\usepackage{indentfirst}
\usepackage{subfig}

\renewcommand\thesection{\Roman{section}}

\begin{document}

\linespread{1.2}

\begin{titlepage}

	\centering
	\includegraphics[width=0.66\textwidth]{elte.jpg}\par\vspace{1cm}
	{\scshape\LARGE ELTE TTK \par}
	\vspace{3cm}
	{\scshape\Large Reaktor üzemeltetési gyakorlat\par}
	\vspace{1cm}
	{\large\itshape Olar Alex\par}
	\vspace{3cm}
	{\large 2018 \par}
	
\end{titlepage}

\begin{abstract}
\par A mérés célja az volt, hogy megismerkedjünk a reaktor üzemeltetéshez szükséges berendezésekkel. Ezek közé tartoznak a biztonságért felelős és mérő műszerek is. A gyakorlat során elindítottuk az önfenntartó láncreakciót és kritikussá tettük a reaktort, valamint vészleállást is kiviteleztünk.
\end{abstract}

\vfill

\tableofcontents

\newpage

\section{Elméleti összefoglaló}

\par A termikus reaktorok $^{235}U$-ös uránnal üzemelnek, amit dúsítani kell, hiszen a természetben ezen urán izotóp részaránya $0.7\%$. Maghasadáskor nagy energiás neutronok keletkeznek, amelyeket termikussá kell tenni, hiszen ekkor a hasító hatáskeresztmetszetük nagyságrendekkel nagyobb. Ezt a célt szolgálja a moderátor anyag, ami a BME tanuló reaktorában $H_{2}O$, azaz víz. Ezen felül a reaktorban grafit reflektorok is találhatóak, amelyek nem moderálnak, hanem a reaktorból kiszökő neutronok számát hivatottak csökkenteni.

\vspace{1cm}

\par Az önfenntartó láncreakció jellemzésére szolgál a kritkusság számszerű jellemzése. Ehhez először bevezetjük a négyfaktor formulát

\begin{equation*}
	k_{eff} = \epsilon \cdot p \cdot f \cdot \eta \cdot P
\end{equation*}

\par ahol $\epsilon$ a gyorsneutronok által okoztt $^{238}U$ hasadások járuléka, $p$ a rezonancia tényező, ami a neutron befogást jellemzi, $f$ a termikus neutronok hasadásba lépő százalékos arányát jellemzi, $\eta$ a termikus neutronhozam, végül $P$ a kilépési tényező, ennek csökkentésé szolgál a grafit reflektor réteg.

\par Ha $k_{eff} < 1$, a rekator szunkritikus, ha nagyobb, akkor szuperkritikus, ha az értéke $1$ akkor a reaktor kritikus.

\par Az ettől való relatív eltérés jellemzésére szolgál a reaktivitás

\begin{equation*}
	\rho = \frac{k_{eff} - 1}{k_{eff}}
\end{equation*}

\par A lancreakció kritikussá tételéhez elengedhetetlenek a késő neutronok, hiszen kockázatos, ha a reaktort már a prompt neutronok is kritikussá tehetik. Ezért érdemes bevezetni a $\frac{\rho}{\beta_{eff}}$ arányt, ahol $\beta_{eff}$ a késő neutronok részaránya. Ennek 'mértékegysége' a $\$$. Az Oktatóreaktor maximális reaktivitása $0.8\$$ körüli.

\vspace{1cm}

\par További biztonsági funkció, hogy a reaktor alulmoderált, azaz ha hirtelen felfutna a láncreakció önmagát leállító módon üzemel, hiszen a felmelegedő moderátor anyag sűrűsége csökken így közvetlenül csökken ennek hatására az effektív sokszorozási tényező, hiszen kevesebb termikus neutron lesz.

\vspace{1cm}

\par Érdemes még megemlíteni, hogy az Oktatóreaktor maximális teljesítménye $100kW$, és még közel $10$ évig biztosan van engedélye az üzemelésre így ezentúl is további diákok látogathatják a létesítményt laborgyakorlat keretében.

\newpage

\section{Mérési feladatok, a mérés menete}

\par A mérés során először meghallgattunk egy előadást a reaktorról, annak történetéről és felépítéséről, valamint átbeszéltük, hogy milyen detektorokkal üzemel a reaktor.

\par 

\section{Összefoglalás}
\par 

\end{document}