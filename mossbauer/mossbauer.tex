\documentclass[a4paper,12pt]{article}
\usepackage[utf8]{inputenc}
\usepackage[T1]{fontenc}
\usepackage[hungarian]{babel}
\usepackage{graphicx}
\usepackage{geometry}
\geometry{a4paper,
		     tmargin = 35mm, 
		     lmargin = 25mm,
		     rmargin = 30mm,
		     bmargin = 30mm}
\usepackage{mathtools}
\usepackage{amsmath}
\usepackage{color}
\usepackage{setspace}
\usepackage{amsmath,amssymb}
\usepackage{float}


\usepackage{indentfirst}
\usepackage{subfig}

\renewcommand\thesection{\Roman{section}}

\begin{document}

\linespread{1.2}

\begin{titlepage}

	\centering
	\includegraphics[width=0.66\textwidth]{elte.jpg}\par\vspace{1cm}
	{\scshape\LARGE ELTE TTK \par}
	\vspace{3cm}
	{\scshape\Large Mössbauer-effektus vizsgálata\par}
	\vspace{1cm}
	{\large\itshape Olar Alex\par}
	\vspace{3cm}
	{\large 2018 \par}
	
\end{titlepage}

\begin{abstract}
\par A mérés célja az volt, hogy megismerkedjünk a legpontosabb energia mérési módszerrel a magfizikában, azaz a Mössbauer-effektussal. Ennek során különböző minták energiaátmeneteit vizsgáltuk, elektromos és mágneses térben.
\end{abstract}

\vfill

\tableofcontents

\newpage

\section{Elméleti összefoglaló}

\par Gerjesztett állapotó atommagok elektromágneses sugárzást bocsáthatnak ki, gamma-sugárzás formájában, amikor alacsonyabb energiaszintre kerülnek. Ennek során a kibocsátott $\gamma$-foton energiája már kisebb, mint egy ugyanilyen mag gerjesztéséhez szükséges energia, hiszen az anyamag visszalökődik. Ahhoz, hogy a kibocsátott foton újra gerjeszteni tudjun a természetes vonalszélességét (energiájának bizonytalanságát) kell növelni. A labor során használt módszer egy hangfalhoz kötötte a mintákat, amiket az állandó gyorsulással rezgetett. A fotonok Doppler-effektusa okozta a vonalkiszélesedést. Így a fotonok már újra el tudtak nyelődni. Az anyagok vizsgálata során elektromos és mágneses térbe is helyezzük a mintákat, vizsgálva például ezáltal a Zeemann-felhasadást.

\vspace{5mm}

\par A Doppler-effektus a sebességgel arányos

\begin{equation*}
	\Delta E = E\sqrt{\frac{c+v}{c-v}} - E \approx E\frac{v}{c}
\end{equation*}

\par Azaz a sebességgel arányos a vonalkiszélesedés. A mérés egy sokcsatornás analizátorra van kötve, amely pedig egy proporcionális kamráról kapja az impulzusokat. Ez az $^{57}Co$ $14.4~keV$-es $\gamma$-fotonjaira érzékeny, amivel a mintát besugározzuk. A mérési berendezés része egy diszkriminátor is, amely a kisebb és nagyobb energiájú fotonokat nem engedi a sok csatornás analizátorra. A spektrum mérése számítógéppel történt a különböző minták behelyezése után.

\section{Mérési eszközök}

\begin{itemize}
\item sokcsatornás anlizátor kártya
\item számítógép
\item diszkriminátor
\item $^{57}Co$, valamint $\text{nátriumprusszid}$,
$\text{lágyvas}$ és $\text{rozsdamentes-acél}$ minták
\end{itemize}

\section{A mérés menete}

\par Először a diszkriminátort állítottuk a megfelelő tartományba, hogy ténylegesen csak a $14.4~keV$-es fotonokat észleljük. 

\section{Összefoglalás}

\par 

\end{document}